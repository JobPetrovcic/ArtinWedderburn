% In this file you should put the actual content of the blueprint.
% It will be used both by the web and the print version.
% It should *not* include the \begin{document}
%
% If you want to split the blueprint content into several files then
% the current file can be a simple sequence of \input. Otherwise It
% can start with a \section or \chapter for instance.

\section{Preliminaries}
  \begin{definition}
      For $a, b \in R$, denote by $aRb$ the set $\{arb| r \in R\}$.
  \end{definition}

  \begin{definition}
    A \textit{left/right ideal} $I$ of a ring $R$ is an additive subgroup such that $rI \subseteq I$ for all $r \in R$ or $Ir \subseteq I$ for all $r \in R$, respectively.
  \end{definition}

  \begin{definition}
    A two-sided ideal is an ideal that is both a left and a right ideal.
  \end{definition}

  \begin{definition}
    A product of (left/right/two-sided) ideals $I$ and $J$ is the ideal generated by the set of all pairwise products of elements of $I$ and $J$.
  \end{definition}

  \begin{definition}
    %\lean{IsPrimeRing}
    
    A ring is prime if from $I * J = 0$ follows $I = 0$ or $J = 0$ for any left ideals $I$ and $J$.
  \end{definition}
  \begin{theorem}
    %\lean{prime_ring_equiv}
    \label{thm:prime_ring_equiv}
    A ring is prime if and only if for all $a, b \in R, aRb = 0$ implies $a = 0$ or $b = 0$.
  \end{theorem}
  \begin{proof}
    $(\Rightarrow)$ Suppose $aRb = 0$. Then $RaRb = 0$, thus by primality $Ra = 0$ or $Rb = 0$. By taking $1 * a$ and $1 * b$ respectively, we get $a = 0$ or $b = 0$.

    $(\Leftarrow)$ Suppose $I * J = 0$. Suppose $a, b \neq 0$ are in $I$ and $J$ respectively. Then $aRb \neq 0$, a contradiction.
  \end{proof}
  \begin{theorem}
    %\lean{prime_ring_equiv'}
    \label{thm:prime_ring_equiv'}
    A ring is prime if and only if for all two-sided ideals $I$ and $J$, $I * J = 0$ implies $I = 0$ or $J = 0$.
  \end{theorem}
  \begin{proof}
    $(\Rightarrow)$ Since two-sided ideals are left ideals and $IJ = 0$ implies $IJ = 0$ as left ideals, we are done.

    $(\Leftarrow)$ Suppose $aRb = 0$. Then $RaRRbR = 0$. By assumption, $RaR = 0$ or $RRbR = 0$. Thus $a = 0$ or $b = 0$ by observing $1 * a * 1$ and $1 * b * 1$. By previous theorem, we are done.
  \end{proof}

  \begin{definition}
    %\lean{def:IsSimpleRing}
    A ring is simple if it has no nontrivial two-sided ideals.
  \end{definition}

  \begin{theorem}
    \label{thm:simple_ring_is_prime}
    %\lean{simple_ring_is_prime}
    A simple ring is prime.
  \end{theorem}
  \begin{proof}
    Suppose $I * J = 0$. If both $I$ and $J$ are nontrivial, they must be equal to $R$ by simplicity. But $R * R \neq 0$, a contradiction.
  \end{proof}

  \begin{definition}
    \label{def:is_semisimple}
    %\lean{IsOrthogonal}
    Two elements $a, b \in R$ are orthogonal if $ab = ba = 0$.
  \end{definition}

\section{Proof of Artin-Wedderburn Theorem for prime and simple rings}
  \begin{theorem}
    \label{thm:one_sub_e_larger_span_on_sub_e_sub_f}
    %\lean{one_sub_e_larger_span_on_sub_e_sub_f}
    If $e, f \in R$ are orthogonal idempotents and $f \neq 0$, then the left ideal generated by $1 - e - f$ is strictly smaller than the left ideal generated by $1 - e$.
  \end{theorem}
  \begin{proof}
    Note that $(1 - e - f)(1 - e) = 1 - e - f$, and hence
    $x(1 - e - f) = x(1 - e - f)(1 - e) \in R(1 - e)$
    for every $x \in R$. This proves that $R(1 - e) \supseteq R(1 - e - f)$.
    
    We have $f = f(1 - e) \in R(1 - e)$, while $f = x(1 - e - f)$ with $x \in R$ implies
    $0 = f(1 - f) = x(1 - e - f)(1 - f) = x(1 - e - f) = f$,
    a contradiction. Therefore, $R(1 - e) \neq R(1 - e - f)$.
  \end{proof}

  From here on, $e, f$ denote orthogonal idempotents in $R$.

  \begin{definition}
    \label{def:corner_ring}
    %\lean{CornerRingSet}
    The set of the corner ring is $eRe$.
  \end{definition}

  \begin{theorem}
    \label{thm:characterization_of_corner_elements}
    An element $x$ is in the set $e R f$ if and only if $x = e x f$.
  \end{theorem}
  \begin{proof}
    $(\Rightarrow)$ Suppose $x \in e R f$. Then $x = e y f$ for some $y \in R$. But then $e x f = e e y f f = e y f = x$.

    $(\Leftarrow)$ Clear.
  \end{proof}
    

  \begin{theorem}
    \label{thm:characterization_of_corner_ring_elements}
    %\lean{x_in_corner_x_eq_e_x_e}
    An element $x$ of $R$ is in the corner ring if and only if $x = e x e$.
  \end{theorem}
  \begin{proof}
    Application of theorem \ref{thm:characterization_of_corner_elements}.
  \end{proof}

  \begin{theorem}
    \label{thm:characterization_of_corner_ring_elements'}
    %\lean{x_in_corner_x_eq_e_y_e}
    An element $x$ of the corner ring is of the form $e y e$ for some $y \in R$.
  \end{theorem}
  \begin{proof}
    Clear from the previous theorem.
  \end{proof}

  \begin{theorem}
    \label{thm:corner_ring_is_ring}
    The corner ring is a (non-unital) subring of $R$. It has its own unit $e$.
  \end{theorem}
  \begin{proof}
    If $a, b \in eRe$, then $a + b = e a e + e b e = e (a + b) e$ so $eRe$ is closed under addition.
    If $a, b \in eRe$, then $a b = e a e e b e  = e a b e$, so $eRe$ is closed under multiplication.
    Distributivity and associativity are inherited from $R$.

    Since $e a = eeae = eae = a = eae = eaee = ae$ for any $a \in eRe$, $e$ is the unit of $eRe$.
  \end{proof}

  \begin{theorem}
    \label{thm:corner_ring_artinian}
    %\lean{corner_ring_artinian}
    If $R$ is artinian, then the corner ring is artinian.
  \end{theorem}
  \begin{proof}
    Let $L_1 \supseteq L_2 \supseteq \ldots$ be a descending chain of left ideals in $eRe$. Then $ R * L_1 * R \supseteq R * L_2 * R \supseteq \ldots$ is a descending chain of left ideals in $R$. Since $R$ is artinian, this chain stabilizes. But then so does $e R L_1 e \supseteq e R L_2 e \supseteq \ldots$. But since $eRL_i = e R e L_i = L_i$, the chain $L_1 \supseteq L_2 \supseteq \ldots$ stabilizes.
  \end{proof}

  \begin{theorem}
    \label{thm:corner_ring_prime}
    %\lean{corner_ring_prime}
    If $R$ is a prime ring, then the corner ring is prime.
  \end{theorem}
  \begin{proof}
    Suppose $a e R e b = 0$ for $a, b \in R$. Then $ae = a = 0$ or $eb = b = 0$ by \ref{thm:prime_ring_equiv}, and by the same theorem, the ring is prime.
  \end{proof}

  \begin{theorem}
    \label{thm:all_left_inv_div_ring}
    If all elements in a ring are left invertible, then the ring is a division ring.
  \end{theorem}
  \begin{proof}
    Let $x \in R$ be arbitrary. Then $yx = 1$ for some $y \in R$. Since $y$ is left invertible, there exists some $z$ such that $zy = 1$. By uniqueness of left and right inverses of $y$ it must hold that $z = x$. Thus $x$ is invertible.
  \end{proof}

  \begin{theorem}
    \label{thm:brauer_lemma}
    %\leanok{minimal_ideal_I_sq_nonzero_exists_idem}
    Suppose $I$ is a minimal (left) ideal of $R$ and $L^2 \neq 0$. Then there exists an idempotent $e \in L$ such that $L = Re$ and $eRe$ is a division ring.
  \end{theorem}
  \begin{proof}
    By assumption, there exists $y \in L$ such that $Ly \neq 0$. By minimality $L = Ly$. Thus, there exists $e \in L$ such that $e y = y$. Let $J \leq L$ be the set of elements in $L$ that annihilate $y$ from the left.
    \begin{claim}
      $J$ is a left ideal of $L$.
    \end{claim}
    \begin{proof}
      Let $a, b \in J$. Then $(a + b) y = a y + b y = 0$, so $(a + b) \in J$. For any $x \in R$, $x a y = 0$ so $xa \in J$. The element $e$ is not in $J$, therefore $J = 0$ by minimality of $L$. 
    \end{proof}
    Rearranging the previous equality, $(e^2 - e) y = 0$ which implies $e^2 = e$, since $e^2 - e$ is in $J = 0$. Clearly $e \neq 0$, and so by minimality (simplicity) $Re = L$.

    Let $a \in eRe$ be non-zero. Then $0 \neq Ra = Reae \leq Re = L$, so $Ra = L$. Thus $e \in Ra$, so $e = r a$ for some $r \in R$. Then $e = e^2 = e r e a$, so $a$ is invertible in $eRe$. We are done by \ref{thm:all_left_inv_div_ring}
    
  \end{proof}

  \begin{theorem}
    %\leanok{artinian_ring_has_minimal_left_ideal}
    (Already proven in Mathlib) An artinian ring has a minimal left ideal.
  \end{theorem}

  \begin{definition}
    \label{def:matrixunits}
    %\leanok{hasMatrixUnits}
    A set $e_{ij}$ for $i, j \in [1, n]$ is a \textit{set of matrix units} of $R$ if 
    $$
      e_{ij}e_{kl} = 
      \begin{cases} 
        e_{il} & j = k \\ 
        0 & \text{otherwise} \end{cases}
    $$ 
    and $\sum_{i=1}^n e_{ii} = 1$.
  \end{definition}

  \begin{theorem}
      \label{thm:ring_with_matrix_units}
      %\lean{ring_with_matrix_units_isomorphic_to_matrix_ring}
      If $R$ has a set of matrix units $e_{ij}$, then $R$ is isomorphic to the ring of $n \times n$ matrices over the corner ring $e_{11}Re_{11}$.
  \end{theorem}
  \begin{proof}
    For $a \in R$, denote $a_{ij} = e_{1i}ae_{j1}$. Then $e_{11}a_{ij}e_{11} = e_{11}e_{1i}ae_{j1}e_{11} = e_{1i}ae_{j1}$ by the property of matrix units. Then, the map $\phi$ claimed to be the isomorphism is $a \mapsto (a_{ij})_{i,j=1}^n$.
    \begin{claim}
      $\phi$ is additive.
    \end{claim}
    \begin{proof}
      For $a, b \in R$, we have:
      $((a + b)_{ij})_{i,j=1}^n = (e_{1i}(a + b)e_{j1})_{i,j=1}^n = (e_{1i}ae_{j1} + e_{1i}be_{j1})_{i,j=1}^n = (a_{ij} + b_{ij})_{i,j=1}^n$
    \end{proof}
    \begin{claim}
      The map is multiplicative.
    \end{claim}
    \begin{proof}
      The $(i,j)$ entry of $\phi(a)\phi(b)$ is equal to 
      $$
        \sum_{k=1}^n e_{1i}ae_{k1}e_{1k}be_{j1} = e_{1i}a \sum_{k=1}^n e_{kk} be_{j1} = e_{1i}abe_{j1},
      $$
      which is the $(i,j)$ entry of $\phi(ab)$. Therefore, $\phi(ab) = \phi(a)\phi(b)$.
    \end{proof}
    \begin{claim}
      The map is injective.
    \end{claim}
    \begin{proof}
      Suppose $a_{ij} = 0$ for all $i, j$. Then $e_{ii}ae_{jj} = e_{1i}a_{ij}e_{j1} = 0$. Therefore, $a = a(\sum_{i=1}^n e_{ii}) = \sum_{i=1}^n ae_{ii} = \sum_{i,j=1}^n e_{ii}ae_{jj} = 0$.
    \end{proof}
    \begin{claim}
      The map is surjective.
    \end{claim}
    \begin{proof}
      Note the $\phi(e_{k1}ae_{1l})_{kl} = e_{1k}e_{k1}a e_{1l}e_{l1} = e_{11}ae_{11}$ and $\phi(e_{k1}ae_{1l})_{ab} = e_{1a}e_{k1}a e_{1l}e_{b1} = 0$ if $a \neq k$ or $b \neq l$, so $\phi(e_{k1}ae_{1l})$ is a matrix whose all entries are zero, except the $k$-th and $l$-th entry is non-zero, and can take arbitrary value in $e_{11}ae_{11}$. By additivity, the map is surjective.
    \end{proof}

  \end{proof}

  \begin{theorem}
    \label{thm:criterion_for_matrix_units}
    %\lean{lemma_2_18}
    If a ring $R$ has a set of pairwise orthogonal idempotents $e_{ii}$ and
    \begin{itemize}
      \item $e_{1i} \in e_{11}Re_{ii}$ for all $i$,
      \item $e_{e1} \in e_{ii}Re_{11}$ for all $i$,
      \item $e_{1i}e_{e1} = e_{11}$ 
      \item $e_{i1}e_{1i} = e_{ii}$ for all $i$,
    \end{itemize}
    then $R$ has matrix units.
  \end{theorem}
  \begin{proof}
    Define $f_{ij} = e_{i1} * e_{1j}$.

    \begin{claim}
      For $i = 1$, we have $f_{1j} = e_{1j}$.
    \end{claim}
    \begin{proof}
      $f_{1j} = e_{11}e_{1j}$. Since $e_{1j} \in e_{11}Re_{jj}$, we have $e_{11}e_{1j} = e_{1j}$ by theorem \ref{thm:characterization_of_corner_elements}.
    \end{proof}
    \begin{claim}
      For $j = 1$, we have $f_{i1} = e_{i1}$ for all $i$.
    \end{claim}
    \begin{proof}
      $f_{i1} = e_{i1}e_{11}$. Since $e_{i1} \in e_{ii}Re_{11}$, we have $e_{i1}e_{11} = e_{i1}$.
    \end{proof}
    \begin{claim}
      $f_{1j} f_{k1} = \delta_{jk} f_{11}$ for all $j, k$
    \end{claim}
    \begin{proof}
      $f_{1j} f_{k1} = e_{11}e_{1j}e_{k1}e_{11} = e_{1j}e_{k1} = e_{11} r e_{jj} e_{kk} r' e_{11} = \delta_{jk} e_{11}$ for some $r, r'$, where the last equality comes from the assumption that the diagonal elements are pairwise orthogonal.
    \end{proof}

    \begin{claim}
      $f_{ij} f_{kl} = \delta_{jk} f_{il}$.
    \end{claim}
    \begin{proof}
      By definition, $f_{ij} f_{kl} = e_{i1}e_{1j} e_{k1}e_{1l} = f_{i1}f_{1j} f_{k1}f_{1l} = \delta_{jk} f_{i1} f_{1l} = \delta_{jk} e_{i1} e_{1l} = \delta_{jk} f_{il}$ by the previous claims.
    \end{proof}
  \end{proof}


  \begin{theorem}
    \label{thm:orthogonal_idempotents_division_ring}
    Let $e, f \in R$ be nonzero orthogonal idempotents and $R$ a prime ring. Also let $eRe$ and $fRf$ be division rings. 

    Then there exist $u, v \in R$ such that $u \in eRf$ and $v \in fRe$ such that $uv = e$ and $vu = f$.
  \end{theorem}
  \begin{proof}
    \begin{claim}
      There exists $a, b \in R$ such that $eafbe \neq 0$.
    \end{claim}
    \begin{proof}
      Suppose $eRf = 0$. By theorem \ref{thm:prime_ring_equiv'}, $eRf = 0$ implies $e = 0$ or $f = 0$, a contradiction. Therefore, there exists $a$ such that $eaf \neq 0$. 

      Suppose $eafRe = 0$. Then $e = 0$ by theorem \ref{thm:prime_ring_equiv'}, a contradiction. Therefore, there exists $b$ such that $eafbe \neq 0$.
    \end{proof}
    Since $eRe$ is a division ring, there exists $c \in R$ such that $(eafbe)(ece) = e$. Let $u = eaf$ and $v = fbece$, which belong to $eRf$ and $fRe$ respectively. Then $uv = eafbece = e$.

    Note that $vu \in fRf$ and that $vuv = ve = v = fv$. Therefore, $(vu - f) v = 0$ 
    \begin{claim}
      $vu = f$.
    \end{claim}
    \begin{proof}
      Suppose not. Then $vu - f \neq 0$, but $vu - f$ is left invertible since $fRf$ is a division ring. Multiplying by the left inverse, we get $v = 0 = fv$, a contradiction with the fact that $uv = e \neq 0$.
    \end{proof}
  \end{proof}

  \begin{theorem}
    \label{thm:orthogonal_idempotents_division_ring_matrix}
    If a prime ring $R$ contains pairwise orthogonal idempotents $e_{ii}$ with sum $1$ and $e_{ii}Re_{ii}$ is a division ring for every $i$, then $R$  is isomorphic to $M_n(e_{11}Re_{11})$.
  \end{theorem}
  \begin{proof}
    Applying the theorem \ref{thm:orthogonal_idempotents_division_ring} for $e_{11}$ and each $e_{ii}$, we define $e_{1i} = u_i$ and $e_{i1} = v_i$ for each $i$ wher $u_i$ and $v_i$ correspond to $u$ and $v$ in the theorem.
    \begin{claim}
      The defined elements satisfy the conditions of theorem \ref{thm:criterion_for_matrix_units}.
    \end{claim}
    \begin{proof}
      By the conclusion of theorem \ref{thm:orthogonal_idempotents_division_ring}, $e_{1i}e_{i1} = e_{ii}$ and $e_{i1}e_{1i} = e_{11}$ for all $i$, and $e_{1i} \in e_{11}Re_{ii}$ and $e_{i1} \in e_{ii}Re_{11}$. The $e_{ii}$ are pairwise orthogonal by assumption.
    \end{proof}
    By theorem \ref{thm:criterion_for_matrix_units}, $R$ has matrix units, and by theorem \ref{thm:ring_with_matrix_units} it is isomorphic to $M_n(e_{11}Re_{11})$.
  \end{proof}

  \begin{theorem}
    \label{thm:idempotents_orthogonal}
    If $e, f \in R$ are idempotents and $f \in (1-e)R(1-e)$ they are orthogonal. Further $fRf = f(1 - e) R (1 - e) f$.
  \end{theorem}
  \begin{proof}
    $f = f(1 - e) + fe = f + fe$. Thus $fe = 0$. Similarly, $ef = 0$. Therefore, $f$ and $e$ are orthogonal.

    Note that $x \in fRf \iff \exists r, x = f r f = f \iff \exists r, x = f(1 - e) r (1 - e) f \iff x \in f(1 - e) R (1 - e) f$. 
  \end{proof}

  \begin{theorem}[Artin Wedderburn for prime rings]
    \label{thm:artin_wedderburn_for_prime}
    %\lean{ArtinWedderburnForPrime}
    If $R$ is a prime ring and artinian, then $R$ is isomorphic to $M_n(D)$ for some division ring $D$.
  \end{theorem}
  \begin{proof}
    Since $R$ is artinian, it contains a minimal nonzero left ideal $L$. If $L^2 = 0$, this would imply by the prime condition that $L = 0$, a contradiction. Therefore, $L^2 \neq 0$. By the Brauer lemma, there exists an idempotent $e \in L$ such that $L = Re_{11}$ and $e_{11}Re_{11}$ is a division ring. By theorem \ref{thm:one_sub_e_larger_span_on_sub_e_sub_f} for $e = 0$ and $f = e_{11}$ we have that $R \supsetneq R(1 - e_{11})$.

    Suppose $e_{11} \neq 1$. Then $(1 - e_{11}) R (1 - e_{11})$ is a nonzero ring. It is also prime and artinian by theorems \ref{thm:corner_ring_prime} and \ref{thm:corner_ring_artinian}. Repeating the argument for this ring, we obtain $e_{22}$ such that $e_{22}(1 - e_{11}) R (1 - e_{11})e_{22}$ is a division ring. Since $e_{22} \in (1 - e_{11}) R (1 - e_{11})$ then must be orthogonal, as by the theorem \ref{thm:idempotents_orthogonal}. Further $R (1 - e_{11}) \supsetneq R (1 - e_{11} - e_{22})$. Repeating this process, we get a sequence of $e_{ii}$ and a sequence of left ideals $R(1 - e_{11} - \ldots - e_{ii})$. By the artinian condition, this sequence must stabilize, so for some $n$, meaning that $\sum_{i = 1}^{n} e_{ii} = 1$. $e_{ii}$ are pairwise orthogonal and are idempotent. Additionally, all $e_{ii} R e_{ii}$ are division rings. By theorem \ref{thm:orthogonal_idempotents_division_ring_matrix}, $R$ is isomorphic to $M_n(e_{11}Re_{11})$.
  \end{proof}

  \begin{theorem}
    \label{thm:artin_wedderburn_for_simple}
    %\lean{ArtinWedderburnForSimple}
    If $R$ is a simple ring, then $R$ is isomorphic to $M_n(D)$ for some division ring $D$.
  \end{theorem}
  \begin{proof}
    Since $R$ is simple, it is prime. By theorem \ref{thm:artin_wedderburn_for_prime}, $R$ is isomorphic to $M_n(D)$ for some division ring $D$.
  \end{proof}


\section{Generalization to semisimple ring}
    In this section, we prove the following result, which clearly generalizes Artin Wedderburn to semisimple rings
    \begin{theorem}
        Let $R$ be a semisimple ring. Then, $R$ is isomorphic to a direct product of simple, artinian rings.
    \end{theorem}
    \begin{proof}
        WLOG, suppose, $R$ is not simple. We know that $R$ is (left) artinian, which is a stronger condition that being (two-sided) artinian. Since it is (two-sided) artinian, it must contain a nontrivial minimal (two-sided) ideal $I$, which is therefore simple. Since $R$ is semisimple, $I$ must be a direct summand of $R$ (AS A LEFT $R$-module). Thus, $R = I \oplus J$ for some (left) ideal $J$. Then $1 = i + j$ for some $i \in I$ and $j \in J$ . Note that $I$ and $J$ are both nontrivial.
    \end{proof}

      \begin{claim}
        $I J = 0$.
      \end{claim}
      \begin{proof}
        Suppose $x \in I J$. Then $x \in I$ since $I$ is a twosided ideal. Also $x \in J$ since $J$ is a left ideal. But then $x = 0$ since $I \cap J = 0$.
      \end{proof}

      \begin{claim}
        $i$ is an idempotent.
      \end{claim}
      \begin{proof}
        $i = i 1 = i(i + j) = i i + i j = i i$ by the previous claim.

      \begin{claim}
        $I I = I$.
      \end{claim}
      \begin{proof}
        By simplicity of $I$, $I I = 0$ or $I$. Since $ i i = i$, the first case is impossible.
      \end{proof}

      \begin{claim}
        $J I = 0$.
      \end{claim}
      \begin{proof}
        Note that $JI$ is spanned by the set of all pairwise products of elements of $J$ and $I$. Since $J$ is a left ideal and $I$ is a two-sided ideal, $JI$ is a two-sided ideal. Then it can be either $0$ or $I$ by simplicity of $I$. 

        Suppose $JI = I$. Then $I = I I = I (J I) = (I J) I = 0 \cdot I = 0$, a contradiction.
      \end{proof}

      \begin{claim}
        $J$ is a two-sided ideal.
      \end{claim}
      \begin{proof}
        We know that it is a left ideal. For arbitrary $x \in R$, write $x = x i + x j$. Let $y \in J$ be arbitrary. Then $y x = y x i + y x j = 0 + y x j \in J$, where $y x i = 0$ since it is in $J I$. Thus $J$ is also a right ideal.
      \end{proof}

      \begin{claim}
        Both $I$ and $J$ are artinian.
      \end{claim}
      \begin{proof}
        They are both submodules of $R$ which is assumed to be artinian. Submodules of artinian modules are artinian. Note that $R$ is artinian since it is semisimple.
      \end{proof}

      \begin{claim}
        $ R = I \times J$ as rings.
      \end{claim}
      \begin{proof}
        Let $x = x_i + x_j$ where $x_i = x i \in I$ and $x j \in J$. Similarly, let $y = y_i + y_j$. Then $x y = x_i y_i + x_i y_j + x_j y_i + x_j y_j = x_i y_i + x_j y_j$ since $x_i y_j = 0$ and $x_j y_i = 0$ by the previous claims. Thus, the map $x \mapsto (x_i, x_j)$ is a ring homomorphism.

        Injective: Suppose $x i = x_j = 0$. Then $x = x 1 = x (i + j) = 0$.

        Surjective: let $(x_i, x_j)\in I \times J$ be arbitrary. Let $x = x_i + x_j$. Note that $x_i = x_i i + x_i j = x_i i$ by orthogonality of $I$ and $J$. Similarly $x_j = x_j j$. Then $x i = (x_i + x_j) i = x_i i = x_i$ and similarly $x j = x_j$. Thus $(x_i, x_j)$ is the image of $x$.
      \end{proof}

      \begin{claim}
        $J$ is (left) semisimple.
      \end{claim}
      \begin{proof}
        A submodule of a semisimple module is semisimple.
      \end{proof}

      Thus, we can repeat the process of splitting $J$ (if it is not simple) into a direct product of simple, artinian rings. Since $R$ is artinian, this process must stabilize, and we get a direct product of simple, artinian rings.
      
      Apply the \ref{thm:artin_wedderburn_for_simple} to each of the simple, artinian rings to get the desired result.
    \end{proof}